%Version 2.1 April 2023
% See section 11 of the User Manual for version history
%
%%%%%%%%%%%%%%%%%%%%%%%%%%%%%%%%%%%%%%%%%%%%%%%%%%%%%%%%%%%%%%%%%%%%%%
%%                                                                 %%
%% Please do not use \input{...} to include other tex files.       %%
%% Submit your LaTeX manuscript as one .tex document.              %%
%%                                                                 %%
%% All additional figures and files should be attached             %%
%% separately and not embedded in the \TeX\ document itself.       %%
%%                                                                 %%
%%%%%%%%%%%%%%%%%%%%%%%%%%%%%%%%%%%%%%%%%%%%%%%%%%%%%%%%%%%%%%%%%%%%%

\documentclass[sn-basic,pdflatex]{sn-jnl}

%%%% Standard Packages
%%<additional latex packages if required can be included here>

\usepackage{graphicx}%
\usepackage{multirow}%
\usepackage{amsmath,amssymb,amsfonts}%
\usepackage{amsthm}%
\usepackage{mathrsfs}%
\usepackage[title]{appendix}%
\usepackage{xcolor}%
\usepackage{textcomp}%
\usepackage{manyfoot}%
\usepackage{booktabs}%
\usepackage{algorithm}%
\usepackage{algorithmicx}%
\usepackage{algpseudocode}%
\usepackage{listings}%
%%%%

%%%%%=============================================================================%%%%
%%%%  Remarks: This template is provided to aid authors with the preparation
%%%%  of original research articles intended for submission to journals published
%%%%  by Springer Nature. The guidance has been prepared in partnership with
%%%%  production teams to conform to Springer Nature technical requirements.
%%%%  Editorial and presentation requirements differ among journal portfolios and
%%%%  research disciplines. You may find sections in this template are irrelevant
%%%%  to your work and are empowered to omit any such section if allowed by the
%%%%  journal you intend to submit to. The submission guidelines and policies
%%%%  of the journal take precedence. A detailed User Manual is available in the
%%%%  template package for technical guidance.
%%%%%=============================================================================%%%%

\usepackage{booktabs}
\usepackage{longtable}
\usepackage{array}
\usepackage{multirow}
\usepackage{wrapfig}
\usepackage{float}
\usepackage{colortbl}
\usepackage{pdflscape}
\usepackage{tabu}
\usepackage{threeparttable}
\usepackage[normalem]{ulem}
\usepackage{threeparttablex}
\usepackage{makecell}
\usepackage{booktabs}
\usepackage{longtable}
\usepackage{array}
\usepackage{multirow}
\usepackage{wrapfig}
\usepackage{float}
\usepackage{colortbl}
\usepackage{pdflscape}
\usepackage{tabu}
\usepackage{threeparttable}
\usepackage{threeparttablex}
\usepackage[normalem]{ulem}
\usepackage{makecell}
\usepackage{xcolor}


\raggedbottom




% tightlist command for lists without linebreak
\providecommand{\tightlist}{%
  \setlength{\itemsep}{0pt}\setlength{\parskip}{0pt}}





\begin{document}


\title[Article Title runing]{Article Title}

%%=============================================================%%
%% Prefix	-> \pfx{Dr}
%% GivenName	-> \fnm{Joergen W.}
%% Particle	-> \spfx{van der} -> surname prefix
%% FamilyName	-> \sur{Ploeg}
%% Suffix	-> \sfx{IV}
%% NatureName	-> \tanm{Poet Laureate} -> Title after name
%% Degrees	-> \dgr{MSc, PhD}
%% \author*[1,2]{\pfx{Dr} \fnm{Joergen W.} \spfx{van der} \sur{Ploeg} \sfx{IV} \tanm{Poet Laureate}
%%                 \dgr{MSc, PhD}}\email{iauthor@gmail.com}
%%=============================================================%%

\author*[1,2]{\pfx{Dr.} \fnm{Leading} \spfx{van} \sur{Author} \sfx{III} \tanm{Poet
Laureate} \dgr{MSc,
PhD}}\email{\href{mailto:abc@def}{\nolinkurl{abc@def}}}

\author[2]{\fnm{Second} \sur{Author} }



  \affil*[1]{\orgdiv{Department}, \orgname{Organization}, \orgaddress{\city{City}, \country{Country}, \postcode{100190}, \state{State}, \street{Street}}}
  \affil*[2]{\orgname{Other Organisation}}

\abstract{\textbf{Purpose}: The abstract serves both as a general
introduction to the topic and as a brief, non-technical summary of the
main results and their implications. The abstract must not include
subheadings (unless expressly permitted in the journal's Instructions to
Authors), equations or citations. As a guide the abstract should not
exceed 200 words. Most journals do not set a hard limit however authors
are advised to check the author instructions for the journal they are
submitting to.

\textbf{Methods:} The abstract serves both as a general introduction to
the topic and as a brief, non-technical summary of the main results and
their implications. The abstract must not include subheadings (unless
expressly permitted in the journal's Instructions to Authors), equations
or citations. As a guide the abstract should not exceed 200 words. Most
journals do not set a hard limit however authors are advised to check
the author instructions for the journal they are submitting to.

\textbf{Results:} The abstract serves both as a general introduction to
the topic and as a brief, non-technical summary of the main results and
their implications. The abstract must not include subheadings (unless
expressly permitted in the journal's Instructions to Authors), equations
or citations. As a guide the abstract should not exceed 200 words. Most
journals do not set a hard limit however authors are advised to check
the author instructions for the journal they are submitting to.

\textbf{Conclusion:} The abstract serves both as a general introduction
to the topic and as a brief, non-technical summary of the main results
and their implications. The abstract must not include subheadings
(unless expressly permitted in the journal's Instructions to Authors),
equations or citations. As a guide the abstract should not exceed 200
words. Most journals do not set a hard limit however authors are advised
to check the author instructions for the journal they are submitting
to.\}}

\keywords{key, dictionary, word}


\pacs[JEL Classification]{D8, H51}
\pacs[MSC Classification]{35A01, 65L10}

\maketitle

TEST

\bibliography{../Data/references.bib}


\end{document}
